\documentclass[12pt]{article}

% Pacotes de Formatação de Texto
\usepackage[portuguese]{babel}
\usepackage[utf8]{inputenc}
\usepackage[a4paper,top=2cm,bottom=2cm,left=2cm,right=2cm]{geometry}
\usepackage{parskip}
\linespread{1.5}

% Pacotes Matemáticos e Gráficos
\usepackage{amsmath}
\usepackage{amssymb}
\usepackage{amsfonts}
\usepackage{cancel}
\usepackage{tikz}

% Outros Pacotes
\usepackage{graphicx}
\usepackage{ascii}
\usepackage{listings}
\usepackage{enumitem}

\usepackage{csquotes}
\usepackage[
    style=authoryear-comp,
    backend=biber,
    maxcitenames=2,
    maxbibnames=99,
    uniquelist=false,
    dashed=false,
    doi=true,
    url=true,
    giveninits=true,
    uniquename=init,
    sorting=nyt,
    language=portuguese,
    natbib=true
]{biblatex}
\addbibresource{referencia.bib}

% Configuração de título e autor
\title{Tarefa EBAC: Questionário \\ \large{Machine Learning 2: Tipos de aprendizados }}
\author{Lucas Amaral Taylor}
\date{\today}

\begin{document}

\maketitle

\begin{enumerate}
    \item \textbf{Diferencie com suas palavras aprendizado supervisionado de aprendizado não supervisionado.}

    Enquanto isso, o \textit{aprendizado supervisionado} é aquele que busca por padrões com um treinamento prévio. Temos que \textit{aprendizado não supervisionado} é aquele que a partir do \textit{dataset} a máquina busca por padrões em dados não rotulados. 
    
    \item \textbf{Pesquise e exemplifique pelo menos mais 2 exemplos de problemas de negócios em que podemos usar aprendizado supervisionado.}

    A partir da referência \citet{eg_aprendizado_supervisionado}, temos que o aprendizado supervisionado pode ser utilizado para:
    \begin{enumerate}
        \item \textbf{Segurança e identificação de fraudes.} O exemplo fornecido pela referência e, também dado em aula, é a identificação de e-mails maliciosos (\textit{spam}). Atualmente, empresas provedoras de e-mail treinam algoritmos para identificação por meio de estrutura do texto, palavras-chave e presença de \textit{links} maliciosos através de uma \textit{dataset} de e-mails classificados ou não como \textit{spam} atribuem um \textit{score} \citep{emails_spam_video}. A identificação destes e-mails é fundamental para evitar o vazamento de dados sensíveis da empresa .

        \item \textbf{Recomendação de produtos com base no comportamento do cliente em um site de compras.} O algoritmo iria por buscar por recomendações personalizadas com grande potencial de compra com base no histórico de visualização e compra de produtos.
        
    \end{enumerate}

    \item \textbf{Pesquise e exemplifique pelo menos mais 1 exemplo de problemas de negócios em que podemos usar aprendizado NÃO supervisionado.}
    
    Utilizando como referência \citet{eg_aprendizado_nao_supervisionado}, temos que podemos utilizar o aprendizado não supervisionado para a identificação de anomalias. O processo de \textit{clustering}  possibilita a detecção de \textit{outliers} e diferenças nos dados. Por exemplo, uma empresa financeira pode usar o aprendizado de máquina não supervisionado para alertá-la sobre cobranças estranhas no cartão da empresa, o que indica transações fraudulentas.
    
    \item \textbf{Por que é essencial que na etapa de teste do aprendizado supervisionado a base de teste não contenha os ``rótulos'' de dados a serem previstos? Justifique com suas palavras.}

    Temos que os ``rótulos'' (\textit{labels}) são retirados na etapa de teste, pois nesta etapa aplicamos uma prova no algoritmo. Na etapa seguinte, chamada avaliação, ``revelamos'' estes resultados e temos a acurácia do modelo e, a partir dela, podemos inferir melhorias e a taxa de erro do modelo.

    Em suma, retiramos os rótulos para testar o quão bem o modelo está funcionando.
    
    \item \textbf{Nas atividades, você tem trabalhado com uma base de dados onde nosso objetivo será de predizer \textit{Churn} dos nossos clientes. Vocês já realizaram a limpeza e tratamento dos dados e já estão realizando a análise exploratória. Qual tipo de aprendizado você acredita ser o ideal para prevermos o \textit{Churn} da base que temos? Justifique com suas palavras.}

    O tipo de aprendizado ideal para prevermos o \textit{Churn} da base que temos é o \textit{não supervisionado}, por dois principais motivos:
    \begin{enumerate}
        \item O \textit{aprendizado não supervisionado} busca identificar padrões com base no \textit{dataset} que temos, como o objetivo é prever o \textit{Churn}, i.e., identificar o possível \textit{Churn} é necessário realizar o processo de \textit{clustering} para encontrar padrões.
        \item O \textit{aprendizado supervisionado} precede uma resposta para a etapa de avaliação. Como queremos identificar um potencial \textit{Churn}, i.e., \textit{predizer}, não temos a resposta necessária para a aplicação do método. 
    \end{enumerate}
\end{enumerate}

\nocite{*}
\printbibliography[title={Referências}, label={sec:bib}]

\end{document}
